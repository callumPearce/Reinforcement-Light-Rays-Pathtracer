% The document class supplies options to control rendering of some standard
% features in the result.  The goal is for uniform style, so some attention 
% to detail is *vital* with all fields.  Each field (i.e., text inside the
% curly braces below, so the MEng text inside {MEng} for instance) should 
% take into account the following:
%
% - author name       should be formatted as "FirstName LastName"
%   (not "Initial LastName" for example),
% - supervisor name   should be formatted as "Title FirstName LastName"
%   (where Title is "Dr." or "Prof." for example),
% - degree programme  should be "BSc", "MEng", "MSci", "MSc" or "PhD",
% - dissertation title should be correctly capitalised (plus you can have
%   an optional sub-title if appropriate, or leave this field blank),
% - dissertation type should be formatted as one of the following:
%   * for the MEng degree programme either "enterprise" or "research" to
%     reflect the stream,
%   * for the MSc  degree programme "$X/Y/Z$" for a project deemed to be
%     X%, Y% and Z% of type I, II and III.
% - year              should be formatted as a 4-digit year of submission
%   (so 2014 rather than the accademic year, say 2013/14 say).


\documentclass[ % the name of the author
                    author={Callum Pearce},
                % the name of the supervisor
                supervisor={Dr. Neill Campbell},
                % the degree programme
                    degree={MEng},
                % the dissertation    title (which cannot be blank)
                     title={How effective are Temporal difference learning methods in reducing the number of zero contribution light paths in Path tracing?},
                % the dissertation subtitle (which can be blank)
                  subtitle={},
                % the dissertation     type
                      type={research},
                % the year of submission
                      year={2019} ]{dissertation}

\begin{document}

% =============================================================================

% This section simply introduces the structural guidelines.  It can clearly
% be deleted (or commented out) if you use the file as a template for your
% own dissertation: everything following it is in the correct order to use 
% as is.

\section*{Prelude}
\thispagestyle{empty}

A typical dissertation will be structured according to (somewhat) standard 
sections, described in what follows.  However, it is hard and perhaps even 
counter-productive to generalise: the goal is {\em not} to be prescriptive, 
but simply to act as a guideline.  In particular, each page count given is
important but {\em not} absolute: their aim is simply to highlight that a 
clear, concise description is better than a rambling alternative that makes
it hard to separate important content and facts from trivia.

You can use this document as a \LaTeX-based~\cite{latexbook1,latexbook2} 
template for your own dissertation by simply deleting extraneous sections
and content; keep in mind that the associated {\tt Makefile} could be of
use, in particular because it automatically executes  to 
deal with the associated bibliography.  

You can, on the other hand, opt {\em not} to use this template; this is a 
perfectly acceptable approach.  Note that a standard cover and declaration 
of authorship may still be produced online via
\[
\mbox{\url{http://www.cs.bris.ac.uk/Teaching/Resources/cover.html}}
\]

% =============================================================================

% This macro creates the standard UoB title page by using information drawn
% from the document class (meaning it is vital you select the correct degree 
% title and so on).

\maketitle

% After the title page (which is a special case in that it is not numbered)
% comes the front matter or preliminaries; this macro signals the start of
% such content, meaning the pages are numbered with Roman numerals.

\frontmatter

% This macro creates the standard UoB declaration; on the printed hard-copy,
% this must be physically signed by the author in the space indicated.

\makedecl

% LaTeX automatically generates a table of contents, plus associated lists 
% of figures, tables and algorithms.  The former is a compulsory part of the
% dissertation, but if you do not require the latter they can be suppressed
% by simply commenting out the associated macro.

\tableofcontents
\listoffigures
\listoftables
\listofalgorithms
\lstlistoflistings

% The following sections are part of the front matter, but are not generated
% automatically by LaTeX; the use of \chapter* means they are not numbered.

% -----------------------------------------------------------------------------

\chapter*{Executive Summary}

\begin{comment}
{\bf A compulsory section, of at most $1$ page} 
\vspace{1cm} 

\noindent
This section should pr\'{e}cis the project context, aims and objectives,
and main contributions (e.g., deliverables) and achievements; the same 
section may be called an abstract elsewhere.  The goal is to ensure the 
reader is clear about what the topic is, what you have done within this 
topic, {\em and} what your view of the outcome is.

The former aspects should be guided by your specification: essentially 
this section is a (very) short version of what is typically the first 
chapter.  Note that for research-type projects, this {\bf must} include 
a clear research hypothesis.  This will obviously differ significantly
for each project, but an example might be as follows:

\begin{quote}
My research hypothesis is that a suitable genetic algorithm will yield
more accurate results (when applied to the standard ACME data set) than 
the algorithm proposed by Jones and Smith, while also executing in less
time.
\end{quote}

\noindent
The latter aspects should (ideally) be presented as a concise, factual 
bullet point list.  Again the points will differ for each project, but 
an might be as follows:

\begin{quote}
\noindent
\begin{itemize}
\item I spent $120$ hours collecting material on and learning about the 
      Java garbage-collection sub-system. 
\item I wrote a total of $5000$ lines of source code, comprising a Linux 
      device driver for a robot (in C) and a GUI (in Java) that is 
      used to control it.
\item I designed a new algorithm for computing the non-linear mapping 
      from A-space to B-space using a genetic algorithm, see page $17$.
\item I implemented a version of the algorithm proposed by Jones and 
      Smith in [6], see page $12$, corrected a mistake in it, and 
      compared the results with several alternatives.
\end{itemize}
\end{quote}
\end{comment}

In the field of Computer Graphics, Path tracing is an algorithm which 
accurately approximates global illumination in order to produce 
photo-realistic images. Path tracing has traditionally been known to 
trade speed for image quality. This is due to the lengthy process of 
finding of accurately finding a pixels colour, whereby many light rays are 
fired through each pixel into scene, then directions for each ray are 
continually sampled until it intersects with a light source. Due to 
this, a variety of Importance sampling algorithms have been invented 
to avoid sampling directions which lead to rays contributing no light 
to the rendered image. The paths formed by sampling rays in these 
directions are known as zero contribution light paths. By not sampling 
zero contribution light paths, it is possible to significantly reduce 
the noise in rendered images using the same number of sampled rays per 
pixel.

Recently a Temporal Difference learning method was used by Nvidia to
achieve impressive results in Importance sampling within a Path tracer.
The algorithm essentially learns which directions light is coming from for
a given point in the scene. It then uses importance sampling to favour shooting
rays those stored directions, reducing the number of zero contribution light paths
sampled. I have further investigated Nvidia's state of the art Expected SARSA
learning algorithm's performance against both a default Path tracer, and my 
newly designed Deep Q-Learning path tracing algorithm. Which leads to my
hypothesis:

\begin{quote}
A deep reinforcement learning Path tracer is able to reduce
the number of zero contribution light paths than an Expected SARSA
Path tracer with the same memory budget.
\end{quote}

\subsection{Plan}
\subsubsection{Breakdown}

\begin{itemize}
\item \textbf{Small Intro}: What is Path tracing (1-2 sentences)? Why is it important (1 sentence)? Seek for real-time ray-tracing (1 sentence). Importance sampling and how temporal difference learning is beginning to be used for importance sampling to avoid sampling rays which do not contribute to the creation of the image(1-2 sentences). 

\item \textbf{Aims \& Objectives}: Evaluate the performance of the state of the art temporal difference learning method compared to a simple path tracer. Furthermore, I present a new deep Q-learning on-line path-tracing scheme and evaluate its performance against the state of the art temporal difference learning scheme and the default path tracer.

\item \textbf{Outcomes}: From my experiments, Temporal difference learning has clear potential to reduce the number of zero contribution light paths sampled by the path tracer, even more so, deep q-learning is shown to outperform temporal difference learning (TODO)

\item \textbf{Main areas of work}: Built a path tracing engine from scratch using only pixel \& basic maths libraries. Researched into efficient light transport simulation. Reimplemented the Irradiance Volume Paper and Nvidia's Learning light the reinforced way paper. Researched into temporal difference learning \& deep reinforcement learning. Developed a new deep q-learning path-tracing algorithm.

\end{itemize}

\subsubsection{Preliminary}
\begin{enumerate}
\item Path tracing is a ray-tracing method for rendering computer generated 
photo-realistic images by accurately approximating global illumination. 
Traditionally it has been thought to trade off rendering speeds with image quality. 

\item The goal is to design and implement modified path-tracing algorithms
 which reduces the number of zero-contribution light paths in its estimation 
 of global illumination. This will lead to less noisy image with the same number
 of samples per pixel. I have integrated different reinforcement learning 
 algorithms into the path-tracing rendering pipeline, which was initially motivated 
 by  Nvidia's promising results when integrating Q-learning % define a light path

\item More specifically, the task of the reinforcement learning AI agent is to learn 
for any given point in the scene the light power contribution from all incident angles. 
This is known as the Irradiance Distribution, the term introduced by (cite The 
Irradiance Volume). It is then possible to importance sample scattering directions 
from the learned irradiance distribution at a given point in the scene to dramatically
reduce the number of rays scattered in directions giving zero-light power 
contribution, also known as zero-contribution light paths. 

\item I have assessed different on-line reinforcement learning techniques for learning
 the irradiance distribution for any point in a scene by comparing metrics such as
  average path length, number of ray paths connecting to a light - generalization

\item I have spent x hours researching into reinforcement learning for Dynamic 
Programming methods, Monte Carlo methods, Temporal Difference methods, Deep
 reinforcement learning methods including monte-carlo and temporal difference 
 learning approaches

\item I have spent x hours researching into the newly emerging field of learning 
light transport

\item I have created my own path-tracing graphics engine which supports naive 
path-tracing, reinforcement learning approach introduced by Ken Dahm, and my 
newly proposed deep reinforcement learning scheme for learning the irradiance 
distribution for any point in the scene
\end{enumerate}

% -----------------------------------------------------------------------------

\chapter*{Supporting Technologies}

{\bf A compulsory section, of at most $1$ page}
\vspace{1cm} 

\noindent
This section should present a detailed summary, in bullet point form, 
of any third-party resources (e.g., hardware and software components) 
used during the project.  Use of such resources is always perfectly 
acceptable: the goal of this section is simply to be clear about how
and where they are used, so that a clear assessment of your work can
result.  The content can focus on the project topic itself (rather,
for example, than including ``I used \mbox{\LaTeX} to prepare my 
dissertation''); an example is as follows:

\begin{quote}
\noindent
\begin{itemize}
\item I used the Java {\tt BigInteger} class to support my implementation 
      of RSA.
\item I used a parts of the OpenCV computer vision library to capture 
      images from a camera, and for various standard operations (e.g., 
      threshold, edge detection).
\item I used an FPGA device supplied by the Department, and altered it 
      to support an open-source UART core obtained from 
      \url{http://opencores.org/}.
\item The web-interface component of my system was implemented by 
      extending the open-source WordPress software available from
      \url{http://wordpress.org/}.
\end{itemize}
\end{quote}

\subsection{Plan}
\begin{enumerate}
\item \verb|SDL2| for displaying the rendered images and saving the images 
as \verb|.bmp| format
\item \verb|OpenGL| mathematics library to support my ray-tracing calculations
 and includes GPU accelerated implementations of each function
\item \verb|CUDA Toolkit 10.1| parallel computing platform for accelerating the 
path tracer implementations
\item \verb|Nvidia 1070Ti| graphics card for rendering images, this was my own 
personal graphics card
\item \verb|Dynet| neural network framework for the implementation of deep 
reinforcement learning within the path tracing rendering pipeline
\end{enumerate}

% -----------------------------------------------------------------------------

\chapter*{Notation and Acronyms}

{\bf An optional section, of roughly $1$ or $2$ pages}
\vspace{1cm} 

\noindent
Any well written document will introduce notation and acronyms before
their use, {\em even if} they are standard in some way: this ensures 
any reader can understand the resulting self-contained content.  

Said introduction can exist within the dissertation itself, wherever 
that is appropriate.  For an acronym, this is typically achieved at 
the first point of use via ``Advanced Encryption Standard (AES)'' or 
similar, noting the capitalisation of relevant letters.  However, it 
can be useful to include an additional, dedicated list at the start 
of the dissertation; the advantage of doing so is that you cannot 
mistakenly use an acronym before defining it.  A limited example is 
as follows:

\begin{quote}
\noindent
\begin{tabular}{lcl}
AES                 &:     & Advanced Encryption Standard                                         \\
DES                 &:     & Data Encryption Standard                                             \\
                    &\vdots&                                                                      \\
${\mathcal H}( x )$ &:     & the Hamming weight of $x$                                            \\
${\mathbb  F}_q$    &:     & a finite field with $q$ elements                                     \\
$x_i$               &:     & the $i$-th bit of some binary sequence $x$, st. $x_i \in \{ 0, 1 \}$ \\
\end{tabular}
\end{quote}

% -----------------------------------------------------------------------------

\chapter*{Acknowledgements}

{\bf An optional section, of at most $1$ page}
\vspace{1cm} 

\noindent
It is common practice (although totally optional) to acknowledge any
third-party advice, contribution or influence you have found useful
during your work.  Examples include support from friends or family, 
the input of your Supervisor and/or Advisor, external organisations 
or persons who  have supplied resources of some kind (e.g., funding, 
advice or time), and so on.

\subsection{Plan}
\begin{enumerate}
\item Carl Henrik Ek - Validating my understanding of deep reinforcement learning
\item Neill Campbell - Deep reinforcement learning strategy
\end{enumerate}

% =============================================================================

% After the front matter comes a number of chapters; under each chapter,
% sections, subsections and even subsubsections are permissible.  The
% pages in this part are numbered with Arabic numerals.  Note that:
%
% - A reference point can be marked using \label{XXX}, and then later
%   referred to via \ref{XXX}; for example Chapter\ref{chap:context}.
% - The chapters are presented here in one file; this can become hard
%   to manage.  An alternative is to save the content in seprate files
%   the use \input{XXX} to import it, which acts like the #include
%   directive in C.

\mainmatter

% -----------------------------------------------------------------------------

\chapter{Contextual Background}
\label{chap:context}

{\bf A compulsory chapter,     of roughly $5$ pages}
\vspace{1cm} 

\noindent
This chapter should describe the project context, and motivate each of
the proposed aims and objectives.  Ideally, it is written at a fairly 
high-level, and easily understood by a reader who is technically 
competent but not an expert in the topic itself.

In short, the goal is to answer three questions for the reader.  First, 
what is the project topic, or problem being investigated?  Second, why 
is the topic important, or rather why should the reader care about it?  
For example, why there is a need for this project (e.g., lack of similar 
software or deficiency in existing software), who will benefit from the 
project and in what way (e.g., end-users, or software developers) what 
work does the project build on and why is the selected approach either
important and/or interesting (e.g., fills a gap in literature, applies
results from another field to a new problem).  Finally, what are the 
central challenges involved and why are they significant? 
 
The chapter should conclude with a concise bullet point list that 
summarises the aims and objectives.  For example:

\begin{quote}
\noindent
The high-level objective of this project is to reduce the performance 
gap between hardware and software implementations of modular arithmetic.  
More specifically, the concrete aims are:

\begin{enumerate}
\item Research and survey literature on public-key cryptography and
      identify the state of the art in exponentiation algorithms.
\item Improve the state of the art algorithm so that it can be used
      in an effective and flexible way on constrained devices.
\item Implement a framework for describing exponentiation algorithms
      and populate it with suitable examples from the literature on 
      an ARM7 platform.
\item Use the framework to perform a study of algorithm performance
      in terms of time and space, and show the proposed improvements
      are worthwhile.
\end{enumerate}
\end{quote}

\subsection{Plan}
\begin{enumerate}
\item Path-tracing in industry/ray-tracing in general, why is it important 
and how is the current field moving. Why should we optimise it algorithmically. 
Why should the reader care about path-tracing? - Usage in films, increasing
 interest for real-time simulations and gaming industry which is worth lots of money

\item High level overview of path-tracing: specifically must explain why it takes 
so long and why we care about the number of samples

\item In the path-tracing algorithm, a single pixel's colour is determined by firing 
multiple rays from the camera, through that pixel into the scene and building a 
colour value estimate for each one, then averaging their values to get the pixels 
colour. Each rays colour estimate is computed by estimating a solution to the recursive
Rendering Equation (cite). The path-tracing algorithms estimate to this solution involves 
scattering the ray around the scene until it intersects with a light source. Therefore, if a
 ray is scattered in a direction with zero-light contribution, but other sampled rays are not,
  a noisy estimate is achieved for the pixel value unless many rays are sampled to 
  reduce the effect of this noise. Therefore, avoiding  scattering rays in directions of 
  zero-light power contribution can reduce the number of samples needed to achieve 
  an accurate estimate of a pixels colour value.

\item Work was primarily motivated by Ken \& Dahms paper for modelling the irradiance
 distribution in order to reduce the number of zero-contribution light transport paths 
 traced. Nvidia are world leaders in GPU manufacturing and drive the computer 
 graphics forward.

\item Literature around efficiently simulating light transport - it's applicability to all 
modern used off-line rendering techniques

\item Aims \& Challenges:

\begin{enumerate}
\item Implementing a path-tracer for diffuse surfaces from scratch using only maths 
and pixel libraries as helper functions which can handle imports of a custom scene
\item Accelerating path-tracer on Cuda to get results in a reasonable time
\item Implementing the irradiance volume data-structure and sampling technique which 
can adapt to any size scene
\item Implementing Ken Dahms proposed path-tracing algorithm with nearest neighbour
 search of KD-Tree on a GPU efficiently 
\item Researching reinforcement learning: TD-Learning \& deep reinforcement learning - 
never been taught before, so self taught with resources on-line
\item Training a network on pre-computed Q values to check if it is possible for a neural
 network to learn the irradiance distribution function for a set of points in a scene
\item Designing an algorithm to integrate deep reinforcement learning into the 
rendering pipeline for a path-tracer
\item Choosing a set of metrics to evaluate the algorithms performances on
\item Accelerating the algorithms via Cuda to run on Nvidia GPU
\end{enumerate}

\end{enumerate}

% -----------------------------------------------------------------------------

\chapter{Technical Background}
\label{chap:technical}

{\bf A compulsory chapter,     of roughly $10$ pages} 
\vspace{1cm} 

\noindent
This chapter is intended to describe the technical basis on which execution
of the project depends.  The goal is to provide a detailed explanation of
the specific problem at hand, and existing work that is relevant (e.g., an
existing algorithm that you use, alternative solutions proposed, supporting
technologies).  

Per the same advice in the handbook, note there is a subtly difference from
this and a full-blown literature review (or survey).  The latter might try
to capture and organise (e.g., categorise somehow) {\em all} related work,
potentially offering meta-analysis, whereas here the goal is simple to
ensure the dissertation is self-contained.  Put another way, after reading 
this chapter a non-expert reader should have obtained enough background to 
understand what {\em you} have done (by reading subsequent sections), then 
accurately assess your work.  You might view an additional goal as giving 
the reader confidence that you are able to absorb, understand and clearly 
communicate highly technical material.

\subsection{Plan}
\begin{enumerate}
\item Define what a ray-tracing rendering algorithm consists of and the difference 
between global and direct illumination. Acknowledge other ray-tracing algorithm l
ike bi-directional path-tracers, Rendermans algorithm, photon mapping. 

\item Define terms like BRDF, radiance, irradiance and the rendering equation

\item Explain the details of the path-tracing algorithm in depth. It should be 
completely clear the relation between path-tracing and the rendering equation. 
It should be clear where the Monte Carlo approach comes in and why 
importance sampling within path-tracing can yield less noisy and more accurate
 results, potentially in the same fixed time-budget

\item Introduce the concept of importance sampling in computing global 
illumination with some early examples of its success, use in industry and recent 
papers on efficient light transport simulation. State the reasoning behind why
it still continues to accurately simulate global illumination, in other words, why
zero-contribution light paths do not contribute to the image.

\item Introduce reinforcement learning: Markov Decision Process, Bellman 
Equation, Temporal Difference Learning and its strong points and weaknesses,
 how does it differ to traditional monte-carlo (might not be relevant). Proved 
 to converge on the true valuation function for a given state-action pair when 
 run infinitely

\item State the derived learning rule supplied by Ken Dahm and visualize the 
matching terms as well as a justification why each parameter matches. What 
is the value and the incentive, diminishing return for rewards far in the future etc

\item State new on-line algorithm proposed by Ken Dahm and details for 
discretizing the state and action space into the Irradiance Volume data-structure
 which was previously introduced 

\item Introduce the concept of deep reinforcement learning, describing how
 DeepMind used the technique for playing Atari games. Given a state give me 
 the state-action values for all actions possible in that state.  Then how we 
 can apply this to our scene to model the state space and continuous.
 
 

\end{enumerate}

% -----------------------------------------------------------------------------

\chapter{Project Execution}
\label{chap:execution}

{\bf A topic-specific chapter, of roughly $15$ pages} 
\vspace{1cm} 

\noindent
This chapter is intended to describe what you did: the goal is to explain
the main activity or activities, of any type, which constituted your work 
during the project.  The content is highly topic-specific, but for many 
projects it will make sense to split the chapter into two sections: one 
will discuss the design of something (e.g., some hardware or software, or 
an algorithm, or experiment), including any rationale or decisions made, 
and the other will discuss how this design was realised via some form of 
implementation.  

This is, of course, far from ideal for {\em many} project topics.  Some
situations which clearly require a different approach include:

\begin{itemize}
\item In a project where asymptotic analysis of some algorithm is the goal,
      there is no real ``design and implementation'' in a traditional sense
      even though the activity of analysis is clearly within the remit of
      this chapter.
\item In a project where analysis of some results is as major, or a more
      major goal than the implementation that produced them, it might be
      sensible to merge this chapter with the next one: the main activity 
      is such that discussion of the results cannot be viewed separately.
\end{itemize}

\noindent
Note that it is common to include evidence of ``best practice'' project 
management (e.g., use of version control, choice of programming language 
and so on).  Rather than simply a rote list, make sure any such content 
is useful and/or informative in some way: for example, if there was a 
decision to be made then explain the trade-offs and implications 
involved.

\section{Example Section}

This is an example section; 
the following content is auto-generated dummy text.
\lipsum

\subsection{Example Sub-section}

\begin{figure}[t]
\centering
foo
\caption{This is an example figure.}
\label{fig}
\end{figure}

\begin{table}[t]
\centering
\begin{tabular}{|cc|c|}
\hline
foo      & bar      & baz      \\
\hline
$0     $ & $0     $ & $0     $ \\
$1     $ & $1     $ & $1     $ \\
$\vdots$ & $\vdots$ & $\vdots$ \\
$9     $ & $9     $ & $9     $ \\
\hline
\end{tabular}
\caption{This is an example table.}
\label{tab}
\end{table}

\begin{algorithm}[t]
\For{$i=0$ {\bf upto} $n$}{
  $t_i \leftarrow 0$\;
}
\caption{This is an example algorithm.}
\label{alg}
\end{algorithm}

\begin{lstlisting}[float={t},caption={This is an example listing.},label={lst},language=C]
for( i = 0; i < n; i++ ) {
  t[ i ] = 0;
}
\end{lstlisting}

This is an example sub-section;
the following content is auto-generated dummy text.
Notice the examples in Figure~\ref{fig}, Table~\ref{tab}, Algorithm~\ref{alg}
and Listing~\ref{lst}.
\lipsum

\subsubsection{Example Sub-sub-section}

This is an example sub-sub-section;
the following content is auto-generated dummy text.
\lipsum

\paragraph{Example paragraph.}

This is an example paragraph; note the trailing full-stop in the title,
which is intended to ensure it does not run into the text.

\subsection{Plan}
\begin{enumerate}
\item Decision on why implementing path-tracer with bare bones tools GLM 
and SDL rather than other packages or frameworks that perform a lot of 
functionality - we needed complete customizability of the algorithm to get
all the data we needed and to modify the pipeline completely. Quick justification 
on why C++ was used

\item Implementing path-tracer and the specific algorithm used with pseudo-code.
With a description of implementing it with CUDA and why the algorithm is 
embarrassingly parralleizable. Present results for different number of samples.

\item Q-value storage implementation using the Irradiance volume structure.
Describe how the space was split up  via sampling irradiance volumes uniformly
over the state space. Visualize the scene discretization with hemispheres and the
voronoi plot.

\item Describe how nearest neighbour search was used to find closest radiance volume
but how this was tricky to convert into CUDA. Describe and state the new algorithm 
proposed by Ken and Dahm with the importance sampling and update steps involving the 
irradiance volumes nearest neighbour search. 

\item Present results for expected SARSA approach and how it compares to the default
path-tracer. Why is it better in terms of number of zero-contribution light paths
included and show how this directly relates to a reduction in noise within the 
image. Also present the fall in avg path length.

\item Introduce state-action space for deep reinforcement learning approach and
how it differs to expected SARSA approach. Introduce reasoning behind wanting to
try the neural network approach vs discretized state space (not continuous)

\item Present findings to see if Q-values  (irradiance distributions) can be learned
by the neural network by using the previous methods converged irradiance volumes outputs
as ground truth and train the network against a subset of them and test the network
against a subset of them.

\item Introduced designed online algorithm to learn the Q-values for any given
state-action pair. Explain discretization of action space. Explain neural network
architecture and why the configuration was chosen. Describe the loss function
and the newly derived update rule.

\item Go through implementation details with GPU and other decisions made

\end{enumerate}

% -----------------------------------------------------------------------------

\chapter{Critical Evaluation}
\label{chap:evaluation}

{\bf A topic-specific chapter, of roughly $15$ pages} 
\vspace{1cm} 

\noindent
This chapter is intended to evaluate what you did.  The content is highly 
topic-specific, but for many projects will have flavours of the following:

\begin{enumerate}
\item functional  testing, including analysis and explanation of failure 
      cases,
\item behavioural testing, often including analysis of any results that 
      draw some form of conclusion wrt. the aims and objectives,
      and
\item evaluation of options and decisions within the project, and/or a
      comparison with alternatives.
\end{enumerate}

\noindent
This chapter often acts to differentiate project quality: even if the work
completed is of a high technical quality, critical yet objective evaluation 
and comparison of the outcomes is crucial.  In essence, the reader wants to
learn something, so the worst examples amount to simple statements of fact 
(e.g., ``graph X shows the result is Y''); the best examples are analytical 
and exploratory (e.g., ``graph X shows the result is Y, which means Z; this 
contradicts [1], which may be because I use a different assumption'').  As 
such, both positive {\em and} negative outcomes are valid {\em if} presented 
in a suitable manner.

\subsection{Plan}
\begin{enumerate}
\item Show for about 4 different scenes the results for a $n$ different numbers of samples; the images, average path length, number of light paths which actually contribute to the image which are sampled between all techniques. I will have to analyse which reduces the number of zero contribution paths the most, but also still assess if the image is photo-realistic.

\item Also analyse default Q-learnings ability on top of expected SARSA

\item Justify reasoning for choosing to analyse Q-Learning, Expected SARSA and DQN (because they have good results for other cases and TD learning fits the online learning procedure)

\item Assess the number of parameters required, configuration is important for these algorithms, if it is very difficult to get right, then the time spent configuring may not be worth it compared to actually rendering the image. E.g. default path-tracing there are not other parameters apart from the number of samples per pixel, expected SARSA requires the user to specify the memory which is allowed to be used by the program, this requires careful consideration, as well as the threshold the distribution cannot fall below, the deep Q-learning algorithm requires less config but potentially different neural network architectures should be investigated to further reduce the number of zero-contribution light paths. 

\item Ease of implementation 

\item Parallelisability of each algorithm, path-tracing is far easier to parallelise as it requires minimal memory accesses by the program to infer pixel values, as opposed to expected SARSA which requires many. Deep-q learning has more customizability in terms of parallelizing (needs more research)

\item Memory usage: Path-tracing is minimal, Expected SARSA is unbounded, Deep Q-Learning is bounded by the size of the neural network, but the memory it requires is still significant (needs more research)

\item DQN vs Expected Sarsa: Do not have to wait for an iteration to begin
 importance sampling on the newly learned Q values for a given point, 
 neural network is continually trained and inferred from. Continuous state 
 space vs discretized required for storage in expected SARSA.
\end{enumerate}

% -----------------------------------------------------------------------------

\chapter{Conclusion}
\label{chap:conclusion}

{\bf A compulsory chapter,     of roughly $5$ pages} 
\vspace{1cm} 

\noindent
The concluding chapter of a dissertation is often underutilised because it 
is too often left too close to the deadline: it is important to allocation
enough attention.  Ideally, the chapter will consist of three parts:

\begin{enumerate}
\item (Re)summarise the main contributions and achievements, in essence
      summing up the content.
\item Clearly state the current project status (e.g., ``X is working, Y 
      is not'') and evaluate what has been achieved with respect to the 
      initial aims and objectives (e.g., ``I completed aim X outlined 
      previously, the evidence for this is within Chapter Y'').  There 
      is no problem including aims which were not completed, but it is 
      important to evaluate and/or justify why this is the case.
\item Outline any open problems or future plans.  Rather than treat this
      only as an exercise in what you {\em could} have done given more 
      time, try to focus on any unexplored options or interesting outcomes
      (e.g., ``my experiment for X gave counter-intuitive results, this 
      could be because Y and would form an interesting area for further 
      study'' or ``users found feature Z of my software difficult to use,
      which is obvious in hindsight but not during at design stage; to 
      resolve this, I could clearly apply the technique of Smith [7]'').
\end{enumerate}

\subsection{Plan}
\begin{enumerate}
\item Summarise contributions:

\begin{enumerate}
\item Implementing a path tracer from scratch to analyse in depth the difficulties and issues that come with Ken Dahm's algorithm. Including memory usage, parallelisation and parameter usage.

\item Analysis of different reinforcement learning approaches pitched together clearly on a variety of scenes

\item Analysis of neural networks ability to learn the irradiance distribution function

\item Online deep-reinforcement learning algorithms effectiveness of learning irradiance distribution function
\end{enumerate}

\item If DQN does not work well provide some further analysis on potential other alternatives which could be used.

\item Future Work: Policy learning to model continuous action \& state space

\item DDQN and other deep reinforcement learning strategies
\end{enumerate}

% =============================================================================

% Finally, after the main matter, the back matter is specified.  This is
% typically populated with just the bibliography.  LaTeX deals with these
% in one of two ways, namely
%
% - inline, which roughly means the author specifies entries using the 
%   \bibitem macro and typesets them manually, or
% - using BiBTeX, which means entries are contained in a separate file
%   (which is essentially a databased) then inported; this is the 
%   approach used below, with the databased being dissertation.bib.
%
% Either way, the each entry has a key (or identifier) which can be used
% in the main matter to cite it, e.g., \cite{X}, \cite[Chapter 2}{Y}.

\backmatter

\bibliography{dissertation}

% -----------------------------------------------------------------------------

% The dissertation concludes with a set of (optional) appendicies; these are 
% the same as chapters in a sense, but once signaled as being appendicies via
% the associated macro, LaTeX manages them appropriatly.

\appendix

\chapter{An Example Appendix}
\label{appx:example}

Content which is not central to, but may enhance the dissertation can be 
included in one or more appendices; examples include, but are not limited
to

\begin{itemize}
\item lengthy mathematical proofs, numerical or graphical results which 
      are summarised in the main body,
\item sample or example calculations, 
      and
\item results of user studies or questionnaires.
\end{itemize}

\noindent
Note that in line with most research conferences, the marking panel is not
obliged to read such appendices.

% =============================================================================

\end{document}
