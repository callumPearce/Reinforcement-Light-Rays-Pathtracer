\documentclass[../dissertation.tex]{subfiles}

\begin{document}

\chapter*{Notation and Acronyms}

\begin{comment}
{\bf An optional section, of roughly $1$ or $2$ pages}
\vspace{1cm} 

\noindent
Any well written document will introduce notation and acronyms before
their use, {\em even if} they are standard in some way: this ensures 
any reader can understand the resulting self-contained content.  

Said introduction can exist within the dissertation itself, wherever 
that is appropriate.  For an acronym, this is typically achieved at 
the first point of use via ``Advanced Encryption Standard (AES)'' or 
similar, noting the capitalisation of relevant letters.  However, it 
can be useful to include an additional, dedicated list at the start 
of the dissertation; the advantage of doing so is that you cannot 
mistakenly use an acronym before defining it.  A limited example is 
as follows:
\end{comment}

\begin{quote}
\noindent

\begin{tabular}{lcl}
SPP &: & Sampled light paths Per Pixel\\
TD learning    &:    & Temporal Difference learning\\
\end{tabular}
\end{quote}

\end{document}