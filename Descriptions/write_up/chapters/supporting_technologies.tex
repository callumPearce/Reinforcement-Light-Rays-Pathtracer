\documentclass[../dissertation.tex]{subfiles}

\begin{document}

\chapter*{Supporting Technologies}

\begin{comment}
{\bf A compulsory section, of at most $1$ page}
\vspace{1cm} 

\noindent
This section should present a detailed summary, in bullet point form, 
of any third-party resources (e.g., hardware and software components) 
used during the project.  Use of such resources is always perfectly 
acceptable: the goal of this section is simply to be clear about how
and where they are used, so that a clear assessment of your work can
result.  The content can focus on the project topic itself (rather,
for example, than including ``I used \mbox{\LaTeX} to prepare my 
dissertation''); an example is as follows:

\begin{quote}
\noindent
\begin{itemize}
\item I used the Java {\tt BigInteger} class to support my implementation 
      of RSA.
\item I used a parts of the OpenCV computer vision library to capture 
      images from a camera, and for various standard operations (e.g., 
      threshold, edge detection).
\item I used an FPGA device supplied by the Department, and altered it 
      to support an open-source UART core obtained from 
      \url{http://opencores.org/}.
\item The web-interface component of my system was implemented by 
      extending the open-source WordPress software available from
      \url{http://wordpress.org/}.
\end{itemize}
\end{quote}
\end{comment}


\begin{enumerate}
\item The \verb|SDL2|  library was used for displaying and saving rendered 
images from my Path tracing engine.

\item The \verb|OpenGL| mathematics library was used to support low level 
operations in my Path tracing engine. It includes GPU accelerated 
 implementations for all of its functions.
 
\item \verb|CUDA Toolkit 10.1| parallel computing platform was used for
 accelerating Path tracing algorithms. This means the \verb|CUDA nvcc|
 compiler must be used to compile our Path tracing engine.

\item All experiments were run on a desktop machine with an
NVIDIA \verb|1070Ti| GPU, Intel \verb|i5-8600K| CPU and $16$GB of RAM.

\item We used the C++ API for the \verb|Dynet| neural network framework 
to implement all of our neural network code, as it is able to be compiled
by the \verb|CUDA| compiler.

\item All graphs created by our own Python scripts using \verb|Matplotlib|
and \verb|NumPy|.
\end{enumerate}

\end{document}