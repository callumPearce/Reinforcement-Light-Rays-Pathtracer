\documentclass[../dissertation.tex]{subfiles}

\begin{document}

\chapter{Critical Evaluation}

\label{chap:evaluation}

Here, a range of techniques are used to evaluate the newly proposed Neural-Q path tracer against the Expected Sarsa path tracer presented in Chapter \ref{chap:td_deep_sampling} on its ability to reduce noise in rendered images produced by Monte Carlo path tracing. The benefits of the Neural-Q path tracer for its lack of hyper parameters which need to be set and its ability to reuse a trained network for a slightly different scene are also discussed. The introduction of deep reinforcement learning for importance sampling light transport paths in Monte Carlo path tracing comes with two important trade-off in practice that limits its current applicability in industry, but presents its clear potential in the long term. The first is the trade-off between being memory bound rendering using the Expected Sarsa path tracer to compute bound using the Neural-Q path tracer. The second is between the significant computational costs that come with the integrated deep reinforcement learning scheme compared to a default forward path tracer for rendering arbitrary scenes.

\section{Experimental Setup}

A path tracing renderer was built from scratch which supported Algorithms \ref{alg:forward_path_tracing}, \ref{alg:expected_sarsa_pathtracer}, \ref{alg:neural_q_pathtracer} and was used to produce all rendered results seen in this thesis. The rendering engine used only OpenGL Mathematics library \cite{glm} for various operations that are common in the rendering pipeline, SDL2 \cite{sdl2} for displaying rendered images, and the Dynet neural network library \cite{dynet} for the Neural-Q path tracer implementation. Algorithms \ref{alg:forward_path_tracing}, \ref{alg:expected_sarsa_pathtracer}, \ref{alg:neural_q_pathtracer} were all accelerated on an Nvidia GPU by using the CUDA Toolkit \cite{cuda} to receive experimental results in an acceptable time. The reasoning for choosing Dynet over more commonly used neural networks libraries such as Tensorflow \cite{tensorflow2015-whitepaper}, was due to its ability to be easily compiled with the CUDA \verb|nvcc| compiler and it had a well documented C++ API. This was a requirement for the Neural-Q path tracer as tracing light paths, ANN inference, and ANN training are all performed on a Nvidia GPU via C++ API calls. All results were produced using a machine with an Intel i5-8600K CPU, Nvidia 1070Ti GPU and 16GB of RAM installed.\\

I developed four different scenes using Maya \cite{maya} and created a custom object importer to import the scenes into my path tracer renderer. A reference image of all four scenes is given in Figure \ref{fig:reference_scenes}, which have been rendered with 4096 sampled light rays per pixel using the default forward path tracing Algorithm \ref{alg:forward_path_tracing}. They are known as reference images as they have minimal visible noise, meaning the Monte Carlo approximations for each pixel's colour value has approximately converged. Due to path tracing accurately modelling physical light transport (see section \ref{sec:mc_pathtracing}), these images are the ideal case for which all other path tracing algorithms should aim to produce as closely as possible with as few samples as possible. 

% Insert reference images of four test scenes

\section{Assessing the reduction in image Noise for Monte Carlo Path Tracing}



\subsection{Quantifying the Reduction in Image Noise}

To quantify the amount noise within images rendered by a default forward path tracer, the Expected Sarsa path tracer, and the Neural-Q path tracer, I will use the Mean Absolute Percentage Error (MAPE) given in Equation \ref{eq:mape} \cite{muller2018neural}.

\begin{equation}
\label{eq:mape}
M = \frac{1}{N} \sum_{i=0}^{N-1} \left| \frac{A_i - F_i}{A_i} \right|
\end{equation}

\noindent
Where:
\begin{conditions}
N & The total number of pixels in the image\\
A_i & The $i$th pixel value in the reference image\\
F_i & The $i$th pixel value in the image whose noise is being quantified\\
\end{conditions}

The MAPE value can therefore be used to quantify the average difference between pixel values from an image rendered by each rendering technique with the same number of sampled light rays per pixel and the reference image. While the MAPE quantifies the overall amount of noise of the image rendered, it does not specify which pixel values in particular have a a high error. I have represented this visually by taking the average squared error across the RGB channels between corresponding pixels in the image being evaluated and the reference image. The image produced highlights failure cases for both the Expected Sarsa and Neural-Q path tracers.

\subsection{Convergence for Learning Incident Radiance}
\begin{itemize}
\item Path length reduction
\item Training curve to show convergence time, can compare the number of zero contribution light paths as when this peaks the algorithm has converged
\end{itemize}

\subsection{Hyper Parameter Tuning}
\begin{itemize}
\item Hyperparams in expected sarsa include what the min of radiance distribution can be set to, the selection of the learning rate $\alpha$, the memory used for the scene...
\item Hyperparams for Neural-Q on the other hand are the epsilon start and epsilon decay
\end{itemize}

\subsection{Reusing Trained ANN for Different Scenes}
\begin{itemize}
\item Expected Sarsa cannot be reused for a slightly different scene whereas the Neural-Q path tracer can 
\item Show the Neural-Q path tracer ability to be reused
\end{itemize}


\section{From Memory Bound to Compute Bound}

Computational resources utilized by the path tracing algorithms introduced is a very important talking point when considering the potential to be integrated into renders used in industry such as \cite{georgiev2018arnold, christensen2018renderman, hyperion}. In particular any bounds imposed on the algorithm from current available hardware are the most important to consider, as rendering algorithms should be designed to give as much power to artists as possible to create and render any arbitrary scene of their choice. A detailed investigation on both highly optimised implementations of the Expected Sarsa and Neural-Q path tracers performance using current available hardware warrants a detailed investigation in itself. Instead for completeness, this section presents a high level analysis of how both memory usage and compute power limits their performance.

\section{Memory Usage}

The default path tracing algorithm uses a minimal amount of memory to calculate the colour estimate of a single light path as it only updates local variables during the computation. Therefore, default path tracing is said to be compute bound, meaning the only way to make the algorithm faster is by providing more compute power. For example parallelizing the rendering process using a GPU with a quicker processor clock speed. However, the performance of the Expected Sarsa algorithm for reducing image noise is bound by the amount of memory the algorithm has available to it. This is due to the requirement of storing the Q-table which holds a discrete number of approximated incident radiance values. The size of memory used by the Q-table is therefore defined by the number of Irradiance Volumes sampled across the scenes geometry, as well as the number incident radiance values stored for the discrete directions $\omega_k$ $\forall i = 1,...,n$  by each Irradiance Volume. The higher the number of sampled Irradiance Volumes the more accurate the incident radiance approximation is to be for a given intersection point as shown in Figure \ref{fig:more_irradiance_volumes}.  By sampling the more Irradiance Volumes, on average the distance from a light path intersection point to the closest Irradiance Volume is likely to be lower. Recall that a nearest neighbour search is performed to calculate to estimate the incident radiance on a point based on the stored values in the closest Irradiance Volume, therefore the closer the Irradiance Volume used for this approximation the more accurate the approximation is likely to be. Moreover, the higher the number of discrete directions$n$ represented by each Irradiance Volume, the more accurate the estimated radiance distribution at the given intersection point will be, which is displayed in Figure \ref{fig:more_directions}. This is because the true radiance distribution at a point is continuous, so the more samples used in the Monte Carlo approximation of this probability density function (see section \ref{sec:td_light_transport}), the lower the error in the approximation due the law of large number in Equation \ref{eq:law_large_numbers}. 

% Stats showing more samples for expected sarsa the better the approximation & images
% Pics for increasing volumes and directions in volumes

The issue is, the amount of memory required to significantly reduce image noise for renders of a scene is dependent on the complexity of the scenes geometry. The more complex the geometry the more sampled Irradiance Volumes which will be required to accurately approximate the radiance distribution at any point in the scene. Meaning, the radiance distributions for two points in the scene which are close to each other are likely to vary far more then that of a scene with simple geometry, as there will a larger number of blockers. %TODO improve this bit 


\subsection{Parallel Processing}

For high quality, high frame rate rendering of scenes, rendering algorithms in general become very computationally demanding \cite{crockett1995parallel}. In order to facilitate this high level of performance, parallel computing must be used to distribute the computational workload. This pattern follows for both scanline renderers and ray-tracing renderers (which path tracing falls into) \cite{alerstam2008parallel, fatahalian2009data}. The default path tracing algorithm in particular is an embarrassingly parallel algorithm \cite{embarissingly_parallelizable}, as the task of finding the colour estimate of a single light path can be performed independently by a single process without any communication to other processes. A simple example of this is given in \cite{accelerated_ray_tracing}, whereby using GPU the process of rendering an image using an algorithm which closely resembles path tracing is split among many cores on a GPU. In industry path tracing algorithms take advantage of parallelism to an unprecedented scale, such as Disney's supercomputer for rendering full-length films using the Hyperion renderer \cite{hyperion}. Therefore, for any modified path tracing algorithm to gain industries attention it needs to be able to take fully advantage of parallel computing hardware.\\

The Expected Sarsa path tracer is no longer as trivial to parallelize as the default forward path tracing algorithm due to the the first addition to the algorithm detailed in section \ref{sec:expected_sarsa_path_tracer}, where the incident radiance estimate for a given entry $Q(x,\omega_k)$ is updated after each intersection point is traced for constructing a light path. To parallelize the algorithm means to create and individual process to construct each light path as part of determining the estimate of a pixel value. However, the Q-table which stores the current approximation of all discrete incident radiance values in the scene is accessed by each thread. Meaning, it is highly possible that two threads will attempt to write to the same location at the same time in the Q-table if they intersect near one another, meaning the update rule provided in Equation \ref{eq:mc_expected_sarsa_td_learning} will not be applied correctly. To remedy this, the update rule can be placed within a critical code section \cite{raynal2012concurrent} to ensure only one thread at a time can calculate and set the updated incident radiance estimate. For my purposes, all code written for parallel execution used the CUDA Toolkit \cite{cuda}, where CUDA atomic operations were used to ensure only one thread at a time can calculate and apply the Q-value update. While the critical section does ensure that the Expected Sarsa path tracer can run in parallel, the performance overhead present for each thread in both waiting to and performing the update rule are added to the path tracing algorithm.\\

As for the Neural-Q path tracer parallel execution is slightly different. Batches of light paths are computed once at a time, where each light path's colour estimate in the batch can be computed in parallel. The batch size should be set to be large enough such that no processors are idle whilst computing a batch. Also, mini-batches which are the same size as the batch of light paths can be used in querying and training the ANN by stochastic gradient descent. However, the computation time to train and evaluate the network for every batch is more expensive than that of updating and looking up the incident radiance values from a Q-table. This performance penalty imposed by the ANN has been found to be very expensive in \cite{keller2019integral, muller2018neural}, and is clear problem which need to be alleviated before neural network importance sampling for Monte Carlo path tracers can be used competitively in industry renderers. Answers to this problem may lie in leveraging the power from recent advancements in hardware such as NVIDIAS TensorCores in Tesla GPUs \cite{tensor_cores} which claims to have 40X higher performance than CPUs for inference on ANNs.

\begin{itemize}
\item Includes compute bottlenecks

\item Here I roughly outline some important implementation details I had to consider which I found should be considered when building a path tracing engine using Algorithm \ref{alg:expected_sarsa_pathtracer}.

\item Firstly, no details regarding how this algorithm is parallelized were given. This is a very important to take into consideration as nearly all computer graphics algorithms leverage the power of Graphical Processing Units (GPUs) for superior speeds in practice. Therefore if the algorithm cannot be parallelized to high degree, it will never be able to perform as well as other existing rendering algorithms.  Every pixel % TODO go in analysis
\end{itemize}


\section{Neural-Q Design Discussion}
Move section down here discussing other methods

\begin{comment}

{\bf A topic-specific chapter, of roughly $15$ pages} 
\vspace{1cm} 

\noindent
This chapter is intended to evaluate what you did.  The content is highly 
topic-specific, but for many projects will have flavours of the following:

\begin{enumerate}
\item functional  testing, including analysis and explanation of failure 
      cases,
\item behavioural testing, often including analysis of any results that 
      draw some form of conclusion wrt. the aims and objectives,
      and
\item evaluation of options and decisions within the project, and/or a
      comparison with alternatives.
\end{enumerate}

\noindent
This chapter often acts to differentiate project quality: even if the work
completed is of a high technical quality, critical yet objective evaluation 
and comparison of the outcomes is crucial.  In essence, the reader wants to
learn something, so the worst examples amount to simple statements of fact 
(e.g., ``graph X shows the result is Y''); the best examples are analytical 
and exploratory (e.g., ``graph X shows the result is Y, which means Z; this 
contradicts [1], which may be because I use a different assumption'').  As 
such, both positive {\em and} negative outcomes are valid {\em if} presented 
in a suitable manner.

\subsection{Plan}

\textbf{Data to collect}
\begin{itemize}

\item Build 4 different scenes:

\begin{itemize}
\item Simple geometry, Indirectly illuminated scene: Here both reinforcement learning methods should perform excellently

\item Simple geometry, Directly illuminated scene: Here all methods should perform well

\item Complex geometry, Indirectly illuminated scene: Can both methods do this - will take a lot of training, deeper NN potentially

\item Complex geometry, Directly illuminated scene: Can both methods do this - will take a lot of training, deeper NN potentially
\end{itemize}

\item Number zero-contribution light paths/ light paths that do not intersect with a with a light after $n$ bounces therefore they become irrelevant for all methods with accumulated frames on the x-axis

\item Variance in points around the room to train network in order to make training batches as varied as possible (this is a weird one, essentially assessing the fact that we do not need a replay buffer).

\item eta-greedy constant for loss curve for training the network \& decaying eta-greedy policy graph for the loss as well

\item Visual representation of Q-values being higher in directions near light source: Map q-values to hemispheres in the scene and get a close up, clearly indicating its ability to sample in the correct direction

\item 1 SPP, 16 SPP, 32 SPP, 64 SPP, 128 SPP, 256 SPP for all three methods on 4 different scenes to evaluate their effectiveness: Assessing accuracy of global illumination approximation

\item Limitations: Number of angles which can accurately be learned by the network, accuracy  needs to be compared with expected SARSA approach for a single radiance volume at a given point in the scene. Size of the scene which can be learnt accurately.

\end{itemize}

\textbf{Preliminary}
\begin{enumerate}
\item Exploration vs Exploitation for both techniques, exploration can yield to better results plus exploitation does not accurately simulate light, relate to the rendering equation and how light works in the physical world.

\item Show for about 4 different scenes the results for a $n$ different numbers of samples; the images, average path length, number of light paths which actually contribute to the image which are sampled between all techniques. I will have to analyse which reduces the number of zero contribution paths the most, but also still assess if the image is photo-realistic.

\item Also analyse default Q-learnings ability on top of expected SARSA

\item Justify reasoning for choosing to analyse Q-Learning, Expected SARSA and DQN (because they have good results for other cases and TD learning fits the online learning procedure)

\item Assess the number of parameters required, configuration is important for these algorithms, if it is very difficult to get right, then the time spent configuring may not be worth it compared to actually rendering the image. E.g. default path-tracing there are not other parameters apart from the number of samples per pixel, expected SARSA requires the user to specify the memory which is allowed to be used by the program, this requires careful consideration, as well as the threshold the distribution cannot fall below, the deep Q-learning algorithm requires less config but potentially different neural network architectures should be investigated to further reduce the number of zero-contribution light paths. 

\item Ease of implementation 

\item Parallelisability of each algorithm, path-tracing is far easier to parallelise as it requires minimal memory accesses by the program to infer pixel values, as opposed to expected SARSA which requires many. Deep-q learning has more customizability in terms of parallelizing (needs more research)

\item Memory usage: Path-tracing is minimal, Expected SARSA is unbounded, Deep Q-Learning is bounded by the size of the neural network, but the memory it requires is still significant (needs more research)

\item DQN vs Expected Sarsa: Do not have to wait for an iteration to begin
 importance sampling on the newly learned Q values for a given point, 
 neural network is continually trained and inferred from. Continuous state 
 space vs discretized required for storage in expected SARSA.
\end{enumerate}

\end{comment}

\end{document}