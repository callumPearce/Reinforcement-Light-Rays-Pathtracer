\documentclass[../dissertation.tex]{subfiles}

\begin{document}

\chapter{Conclusion}
\label{chap:conclusion}

{\bf A compulsory chapter,     of roughly $5$ pages} 
\vspace{1cm} 

\noindent
The concluding chapter of a dissertation is often underutilised because it 
is too often left too close to the deadline: it is important to allocation
enough attention.  Ideally, the chapter will consist of three parts:

\begin{enumerate}
\item (Re)summarise the main contributions and achievements, in essence
      summing up the content.
\item Clearly state the current project status (e.g., ``X is working, Y 
      is not'') and evaluate what has been achieved with respect to the 
      initial aims and objectives (e.g., ``I completed aim X outlined 
      previously, the evidence for this is within Chapter Y'').  There 
      is no problem including aims which were not completed, but it is 
      important to evaluate and/or justify why this is the case.
\item Outline any open problems or future plans.  Rather than treat this
      only as an exercise in what you {\em could} have done given more 
      time, try to focus on any unexplored options or interesting outcomes
      (e.g., ``my experiment for X gave counter-intuitive results, this 
      could be because Y and would form an interesting area for further 
      study'' or ``users found feature Z of my software difficult to use,
      which is obvious in hindsight but not during at design stage; to 
      resolve this, I could clearly apply the technique of Smith [7]'').
\end{enumerate}

\subsection{Plan}
\begin{enumerate}

\item Problem:

\item Summarise contributions to solving problem:
\begin{enumerate}
\item Extended temporal difference learning for the Incident Radiance function to the continuous spatial domain 

\item Proved it is possible for a 2 layer ANN to learn an accurate approximation of the Incident Radiance Function for a variety of scenes by the development of the new Neural-Q path tracing algorithm

\item Presented the weakness of Expected Sarsa when learning the incident radiance distribution on a point compared to the Neural-Q's smoother and more accurate approximation.

\item Presented that the Neural-Q algorithm is better able to reduce image noise for certain scenes compared to the Expected Sarsa algorithm proposed by NVIDIA using a smaller amount of memory and simpler hyperparameter tuning. Whilst sacrificing compute speed, limiting the algorithms current applicability to industry.
\end{enumerate}

\item Discussion of main findings and drawbacks:

\item Future work:

\begin{itemize}
\item If DQN does not work well provide some further analysis on potential other alternatives which could be used.

\item Future Work: Policy learning to model continuous action \& state space

\item DDQN and other deep reinforcement learning strategies
\end{itemize}
\end{enumerate}

\end{document}