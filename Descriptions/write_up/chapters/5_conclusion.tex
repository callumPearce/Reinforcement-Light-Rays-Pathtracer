\documentclass[../dissertation.tex]{subfiles}

\begin{document}

\chapter{Conclusions}
\label{chap:conclusion}

The objectives set out by this thesis were to find a way to accurately learn the incident radiance function over the continuous set of locations in a scene. Then to design an algorithm which used this approximation for importance sampling directions to continue light paths in as part of Monte Carlo path tracing to reduce image noise. This algorithm then needs to be compared against an existing state of the art method which instead discretizes the continuous set of possible locations in the scene to approximate the incident radiance function. With the new algorithm designed and evaluated, the question of is learning the incident radiance over the continuous set of possible locations in a scene beneficial for importance sampling in Monte Carlo path tracing, when compared to existing approaches in any way? This chapter describes our achievement of these goals and in turn the contributes we have added to the field of computer graphics and efficient numerical integration. Followed by a discussion on the future research opportunities which have arisen due to the conclusions made.

\section{Wider Motivation and the Problem}

Monte Carlo path tracing is a method used in Computer Graphics capable of producing photo-realistic images. Traditionally, it is known to trade off rendering time for the superior image quality it can produce, however increasing compute power, innovative algorithms, and post processing have been shown to accelerate the speed of the algorithm by a significant amount. This combined with the lack of set-up required to render any arbitrary scene has led to a large resurgence in the algorithms use in industry \cite{keller2015path}. A larger goal held across industry at this point is to achieve real-time rendering for methods like path tracing which accurately simulate light transport \cite{akenine2018real}, but more work is required to make this a reality .

Monte Carlo path tracing uses Monte Carlo integration to approximate the true pixel value for all pixels in an image to render a scene. Where a sample is represented by a single light path. Therefore, importance sampling light paths for Monte Carlo path tracing has the potential to significantly improve every pixels colour estimate within the image, while using the same number of SPP. But the task of finding correct information to importance sample light path directions with is a difficult one, and there have been many different approaches for doing so \cite{vorba2014line, muller2017practical, dahm2017learning}. NVIDIA's technique for this is what we refer to as the Expected Sarsa path tracer, which introduced the application of reinforcement learning for approximating the incident radiance function, as well as a new path tracing algorithm to use this approximation for importance sampling light path directions \cite{dahm2017learning}. Much like the other existing methods, their tabular TD-learning approach required a discretization of locations in the scene to approximate the incident radiance function, whereas the scene in actual fact contains an infinite number of discrete positions. This raised the question that is it possible to learn the incident radiance function over the continuous set of locations in the scene using a function approximator instead? Also, is this advantageous for reducing noise in images produced by Monte Carlo path tracing compared to existing methods?

\section{Summary of Contributions}

We have introduced an ANN loss function based on Deep Q-learning \cite{mnih2013playing}, along with an ANN architecture for learning the incident radiance function $L_i(x, \omega)$ \cite{kajiya1986rendering} for an arbitrary scene. A novel part of the introduced architecture is instead of passing a single position to the ANN as input, all vertices converted into a coordinate system relative to the point are passed in. Without this, the ANN was not able to learn the incident radiance function for even the most simple scenes. To make use of the proposed ANN and loss function, a new path tracing algorithm known as the Neural-Q path tracer has been developed. Neural-Q uses the proposed learning rule to efficiently train the ANN online during the rendering process. Much like Expected Sarsa, the approximation of the incident radiance function is then used to importance sample directions to continue light paths in, and to calculate probability density function needed for correctly evaluating Monte Carlo path tracing. As more light paths are sampled in the rendering process, the accuracy of the approximated incident radiance function improves, leading to more efficient importance sampling of directions to continue light paths in. In turn, the variance in the approximation of pixel values during Monte Carlo path tracing using the same number of SPP is reduced. Directly reducing image noise. 

We have also introduced and compared the Expected Sarsa path tracer to our Neural-Q path tracer as a means of assessing the extension of approximating the incident radiance over the set of continuous locations in a scene. In doing so, we found this extension brought various advantages for importance sampling Monte Carlo path tracing. These include; improved incident radiance function approximation for scenes with more complex geometry, leading to noise reduction in images produced by Monte Carlo path tracing. As well as, Improved memory scaling for rendering scenes with more complex geometry, and simpler hyperparemeters for tuning. The investigation also uncovered that the Expected Sarsa algorithms applicability for rendering in industry is limited by the memory required to store the Q-table for approximating the incident radiance function. Whereas, the Neural-Q algorithm is limited by the time taken to evaluate a forward pass on an ANN. This conclusion was also reached by two studies published during the execution of this thesis \cite{keller2019integral, muller2018neural}.

\section{Discussion}

In this section the key conclusions made by our work are summarised. 

\subsubsection*{Reducing noise in Monte Carlo path tracing renders}
The results presented so far for the newly developed Neural-Q algorithm are promising. It is able to significantly reduce noise for scenes rendered by Monte Carlo path tracing using the same number of SPP. Furthermore, it is able to reduce image noise more compared to that of the state of the art Expected Sarsa path tracer for scenes with more complex geometry. While at the same time using a smaller, constant amount of memory. Hyperparamter tuning was also a much simpler task, as determining the parameters for the decaying $\epsilon$-greedy strategy used by Neural-Q for the scenes tested was found to be far easier and quicker task, compared to deciding the amount of memory the Expected Sarsa algorithm requires to significantly reduce image noise. These factors combined with Neural-Q's online training to progressively reduce image noise across accumulated rendered frames for a scene, make it a promising addition to the field of computer graphics. However, it is important to realise that newly proposed rendering algorithms should be tested for a broad range of scenes to fully assess which ones they apply best to. The Neural-Q algorithm and ANN architecture we propose is no exception to this. While we have shown it is able to perform well for a selection of different scenes exhibiting different properties, scenes used in film and game production are yet to be experimented with for both the Neural-Q and Expected Sarsa path tracers.

\subsubsection*{Improving the approximation of the incident radiance function}
The reasoning for the reduction in noise was found to be the more accurate approximation made by the Neural-Q algorithm for the incident radiance function $L_i(x, \omega)$. This conclusion was reached by both further analysing the noise present in images rendered by the Expected Sarsa method and visualising the incident radiance distribution on three different points in a simple scene. Expected Sarsa's poorer performance for renders was found to be a product of the more inaccurate incident radiance function approximation made compared to that of the Neural-Q path tracers. This was related back to the theory of Monte Carlo importance sampling, which also revealed why there were a large number of fireflies present in the Expected Sarsa renders. 

\subsubsection*{Memory versus compute time}
While both the Neural-Q and Expected Sarsa path tracers are able to significantly reduce noise in images rendered by Monte Carlo path tracing, they both come with costs. The Expected Sarsa was found to be limited in its ability to reduce noise for scenes by the amount of memory available in the underlying the hardware to store the Q-table for approximating the incident radiance function. The real concern from this was that more complex scenes required more memory to receive a significant reduction in image noise. Meaning, Expected Sarsa may not be applicable to scenes used in the production of films which include millions of polygons, due to the memory requirements to store the Q-table being far too large.

On the other hand, the Neural-Q algorithms requirement to frequently evaluate forward passes on an ANN for tracing every light path is very computationally expensive. In fact so much so, new purpose built hardware offered for quick ANN inference running per thread on a GPU will likely be required to have any hope of using Neural-Q in industry.

\subsubsection*{Neural Importance Sampling}
The benefits which are received by using ANNs for sampling directions to continue light paths in for Monte Carlo path tracing are now clear. Not only does this present an addition to the field of computer graphics, but also presents a clear application of using neural networks for importance sampling in Monte Carlo methods, otherwise known as neural importance sampling \cite{keller2019integral}. Meaning, other methods which require Monte Carlo integration to numerically solve integrals may benefit from an improvement in efficiency by adapting the approach outlined here to their problem. In fact, this is a hot topic in computer graphics right now, where leaders in the computer graphics industry including Disney and NVIDIA are making serious progress in the application of neural importance sampling to Monte Carlo rendering techniques \cite{keller2019integral, muller2018neural}. But, to the best of our knowledge, the Neural-Q algorithm and ANN architecture we introduced is the first which uses a shallow ANN for importance sampling light path directions in Monte Carlo rendering.

Assessing the data collected in recent investigations, it seems as though approximating the incident radiance using Monte Carlo integration with neural importance sampling, rather than using an ANNs approximation of the incident radiance function to directly infer a pixels colour value produces higher quality images \cite{zheng2018learning, keller2019integral, muller2018neural}. Hence, it is an area which we are likely to see a significant amount of research in within the near future.

\section{Future Work}

With the introduction of the new loss function and Neural-Q path tracer, as well as recent studies on neural importance sampling published during the execution of this thesis, many areas of future research have opened up. In this section we present the most interesting areas which we believe will lead to answers for the most important questions currently surround neural importance sampling techniques for Monte Carlo path tracing. 

\subsubsection{Scaling up Neural-Q}

We consider this to be most important area to further extend the research completed by this thesis. There are three areas we propose that the Neural-Q algorithm should be scaled up:

\begin{enumerate}
\item \textbf{Testing high polygon count scenes} - For the Neural-Q path tracer to be adopted in industry it needs to be tested on scenes with potentially millions of polygons. If Neural-Q is able to continue to significantly reduce the noise in rendered images of such scenes, then it is only a question of how to speed the algorithm up to make it competitive with existing algorithms. However, we believe it is more likely further development will have to be done on the ANN architecture used by the Neural-Q path tracer to learn the incident radiance function for a very complex scene. Our reasoning behind this thought is due to the complexity of the ANN architecture described in \cite{muller2018neural} to model scenes with complex geometry. However, our idea of using the vertices converted into a coordinate system centred around the input position may reduce the need for this complexity.

\item \textbf{Accelerating Neural-Q on optimised hardware} - The experiments conducted in this thesis for the Neural-Q path tracer used an NVIDIA 1070Ti GPU, which is not even the latest series of commercial GPUs released by NVIDIA with the introduction of their new 20 series GPUs \cite{nvidia_turing_architecture_whitepaper_2018}. Therefore, we believe it is necessary to test Neural-Q path tracer on a variety of the latest available graphics cards, prioritising those with NVIDIA Tensor Cores \cite{tensor_cores} in hope to significantly speed up inference during light path construction. Note, these tests should also be run on a more optimised path tracing engine then the one developed for this project, including common features such as bounding boxes \cite{boulos2005notes} and practices like aligned memory accesses. Comparisons should then be made against existing methods for importance sampling in Monte Carlo path tracing, such as the Expected Sarsa algorithm and many others.

\item \textbf{Integrating Neural-Q into a production path tracer} - This point relies on the two previous ones being completed and receiving satisfactory data that proves Neural-Q can be applied to industry for rendering. By integrating Neural-Q to a production renderer, the algorithm can be assessed on rendering films and whether artists find it to be helpful or taxing to their work. This will also check if the algorithm combines well with post processing used in production renderer pipelines \cite{georgiev2018arnold, christensen2018renderman}. For example image denoising by ANNs allowing the use of far less samples in Monte Carlo path tracing to begin with \cite{bako2017kernel, chaitanya2017interactive}.
\end{enumerate}

\subsubsection{Investigating the relationship between scene geometry and ANN requirements}

From the results of the Neural-Q algorithm outperforming the Expected Sarsa method for scenes with more complex geometry, yet comparable or worse for those with simpler geometry, it is clear more research must be done to investigate how ANN architecture relates to the accuracy of the approximated incident radiance function for a scene. This is no easy task due to the variety of options to explore deep learning. Potentially assessing other function approximators such as Gaussian Processes for learning this function instead may potentially be profitable. However, we believe due to the abundance of data available in by continually sampling light paths, ANNs are likely to produce the best results. 

\subsubsection{Neural-Q versus Disney's Neural Path Guider}

The only other work to our knowledge which uses ANNs for importance sampling light path directions in Monte Carlo path tracing is that of Disney Research which was published during the execution of this thesis \cite{muller2018neural}. As previously described, the ANN framework they introduced known as NPG also learns online, hence the Neural-Q algorithm and NPG can be tested against one another for their performance in various aspects. We believe NPG is likely to be more successful in reducing noise for an arbitrary scene, however it will be slower in doing so due to the smaller ANN architecture introduced by Neural-Q. We believe this is an important area to assess, as there are no current studies published regarding the comparison of multiple neural importance sampling schemes to  the best of our knowledge.

\subsubsection{Further investigate Deep Reinforcement learning techniques}

The field of Deep Reinforcement learning also has plenty to offer for the further development of the Neural-Q path tracer. Other algorithms such double deep q-learning (DDQN) which has been shown to improve the efficiency of the learning process, which could potentially be applied to the Neural-Q path tracer for quicker convergence on a good approximation of the incident radiance function. Semi gradient one-step Expected Sarsa could be used to derive an alternative loss function which more closely resembles the rendering function \cite{}, by taking a similar approach to that derived in \icte{newdahm}. Also, a replay buffer may also be applied to further reduce the correlation between consecutive training updates when constructing a batch of light paths.


\begin{comment}

{\bf A compulsory chapter,     of roughly $5$ pages} 
\vspace{1cm} 

\noindent
The concluding chapter of a dissertation is often underutilised because it 
is too often left too close to the deadline: it is important to allocation
enough attention.  Ideally, the chapter will consist of three parts:

\begin{enumerate}
\item (Re)summarise the main contributions and achievements, in essence
      summing up the content.
\item Clearly state the current project status (e.g., ``X is working, Y 
      is not'') and evaluate what has been achieved with respect to the 
      initial aims and objectives (e.g., ``I completed aim X outlined 
      previously, the evidence for this is within Chapter Y'').  There 
      is no problem including aims which were not completed, but it is 
      important to evaluate and/or justify why this is the case.
\item Outline any open problems or future plans.  Rather than treat this
      only as an exercise in what you {\em could} have done given more 
      time, try to focus on any unexplored options or interesting outcomes
      (e.g., ``my experiment for X gave counter-intuitive results, this 
      could be because Y and would form an interesting area for further 
      study'' or ``users found feature Z of my software difficult to use,
      which is obvious in hindsight but not during at design stage; to 
      resolve this, I could clearly apply the technique of Smith [7]'').
\end{enumerate}

\subsection{Plan}
\begin{enumerate}

\item Summarise contributions to solving problem:
\begin{enumerate}
\item Extended temporal difference learning for the Incident Radiance function to the continuous spatial domain 

\item Proved it is possible for a 2 layer ANN to learn an accurate approximation of the Incident Radiance Function for a variety of scenes by the development of the new Neural-Q path tracing algorithm

\item Presented the weakness of Expected Sarsa when learning the incident radiance distribution on a point compared to the Neural-Q's smoother and more accurate approximation.

\item Presented that the Neural-Q algorithm is better able to reduce image noise for certain scenes compared to the Expected Sarsa algorithm proposed by NVIDIA using a smaller amount of memory and simpler hyperparameter tuning. Whilst sacrificing compute speed, limiting the algorithms current applicability to industry.
\end{enumerate}

\item Discussion of main findings and drawbacks:

\item Future work:

\begin{itemize}
\item If DQN does not work well provide some further analysis on potential other alternatives which could be used.

\item Future Work: Policy learning to model continuous action \& state space

\item DDQN and other deep reinforcement learning strategies
\end{itemize}
\end{enumerate}

\end{comment}

\end{document}