\documentclass[../dissertation.tex]{subfiles}

\begin{document}

\chapter*{Executive Summary}

\begin{comment}
{\bf A compulsory section, of at most $1$ page} 
\vspace{1cm} 

\noindent
This section should pr\'{e}cis the project context, aims and objectives,
and main contributions (e.g., deliverables) and achievements; the same 
section may be called an abstract elsewhere.  The goal is to ensure the 
reader is clear about what the topic is, what you have done within this 
topic, {\em and} what your view of the outcome is.

The former aspects should be guided by your specification: essentially 
this section is a (very) short version of what is typically the first 
chapter.  Note that for research-type projects, this {\bf must} include 
a clear research hypothesis.  This will obviously differ significantly
for each project, but an example might be as follows:

\begin{quote}
My research hypothesis is that a suitable genetic algorithm will yield
more accurate results (when applied to the standard ACME data set) than 
the algorithm proposed by Jones and Smith, while also executing in less
time.
\end{quote}

\noindent
The latter aspects should (ideally) be presented as a concise, factual 
bullet point list.  Again the points will differ for each project, but 
an might be as follows:

\begin{quote}
\noindent
\begin{itemize}
\item I spent $120$ hours collecting material on and learning about the 
      Java garbage-collection sub-system. 
\item I wrote a total of $5000$ lines of source code, comprising a Linux 
      device driver for a robot (in C) and a GUI (in Java) that is 
      used to control it.
\item I designed a new algorithm for computing the non-linear mapping 
      from A-space to B-space using a genetic algorithm, see page $17$.
\item I implemented a version of the algorithm proposed by Jones and 
      Smith in [6], see page $12$, corrected a mistake in it, and 
      compared the results with several alternatives.
\end{itemize}
\end{quote}
\end{comment}

In the field of Computer Graphics, Monte Carlo path tracing is an algorithm which is capable of rendering photo-realistic images. Traditionally Monte Carlo path tracing has been thought to trade quick render times for superior image quality. This is due to the lengthy process of accurately finding each pixel's colour using Monte Carlo integration. At a high level, many light rays are traced from a camera, through each pixel and into scene, where each ray intersects with a point on a surface. Then, a new direction is sampled for each ray to continue their path in, where they intersect with another surface. A direction to continue the path of a ray in is continually sampled until it intersects with a light source. Each one of these rays traced through a pixel, around the scene and to a light source are a sample used to solve an integral by Monte Carlo integration which determines the pixel's colour value.

As path tracing computes each pixel value in an image by Monte Carlo integration, importance sampling can be applied, whereby directions which are considered to be more 'important' for determining a pixel's colour value have a higher probability of being sampled. Therefore, if we have a good idea of which directions are more important to continue rays in, each one of our sampled rays per pixel will contribute more to the pixel's value, meaning fewer samples will be required to accurately approximate the pixels colour value via Monte Carlo integration. In other words, the noise present in each pixel's colour value will be lower by using importance sampling with the same number of samples per pixel than without. If the noise in each pixel's colour estimate is lower, then the noise across the rendered image by path tracing will be lower, which ultimately improves the quality of the image at the same computational cost.

In order to determine which directions are more important to continue a ray in at a given intersection point, a reinforcement learning method, specifically temporal difference learning method has been used in a new path tracing algorithm to learn what is known as the incident radiance function. The approximation of this function made by the algorithm is used to importance sample directions to continue light paths in, which successfully reduces image noise in Monte Carlo path tracing. However, the algorithm approximates the function at a discrete set of locations within the scene, where in actual fact the scene is a continuous 3-dimensional space. Therefore, we have tested the following research hypothesis:

\begin{quote}
Learning the incident radiance function for the continuous set of locations in a scene is more advantageous for importance sampling in Monte Carlo path tracing, compared to learning the function for a discrete set of locations.
\end{quote}

\noindent
In doing so, our main areas of work include:

\begin{itemize}
\item Built a path tracing engine from scratch which supports all path tracing algorithms we will introduce and the code for assessing them. This sums to over 7000 lines of C++/CUDA C code.

\item Spent 50 hours researching into the field of efficient light transport simulation for path tracing techniques.

\item Spent 130 hours researching into Reinforcement Learning and Deep Reinforcement Learning.

\item Implemented the state of the art path tracing algorithm proposed by NVIDIA in \cite{dahm2017learning}  in which the incident radiance function is approximated for a discrete set of locations in a scene.

\item Designed and implemented our newly proposed Neural-Q path tracer which approximates the incident radiance function for the continuous set of locations in a scene.

\item Assessed the performance of the newly designed Neural-Q path tracer against NVIDIA's path tracer for various metrics, including the accuracy of the approximated incident radiance function.
\end{itemize}


\end{document}