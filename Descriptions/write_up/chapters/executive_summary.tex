\documentclass[../dissertation.tex]{subfiles}

\begin{document}

\chapter*{Executive Summary}

\begin{comment}
{\bf A compulsory section, of at most $1$ page} 
\vspace{1cm} 

\noindent
This section should pr\'{e}cis the project context, aims and objectives,
and main contributions (e.g., deliverables) and achievements; the same 
section may be called an abstract elsewhere.  The goal is to ensure the 
reader is clear about what the topic is, what you have done within this 
topic, {\em and} what your view of the outcome is.

The former aspects should be guided by your specification: essentially 
this section is a (very) short version of what is typically the first 
chapter.  Note that for research-type projects, this {\bf must} include 
a clear research hypothesis.  This will obviously differ significantly
for each project, but an example might be as follows:

\begin{quote}
My research hypothesis is that a suitable genetic algorithm will yield
more accurate results (when applied to the standard ACME data set) than 
the algorithm proposed by Jones and Smith, while also executing in less
time.
\end{quote}

\noindent
The latter aspects should (ideally) be presented as a concise, factual 
bullet point list.  Again the points will differ for each project, but 
an might be as follows:

\begin{quote}
\noindent
\begin{itemize}
\item I spent $120$ hours collecting material on and learning about the 
      Java garbage-collection sub-system. 
\item I wrote a total of $5000$ lines of source code, comprising a Linux 
      device driver for a robot (in C) and a GUI (in Java) that is 
      used to control it.
\item I designed a new algorithm for computing the non-linear mapping 
      from A-space to B-space using a genetic algorithm, see page $17$.
\item I implemented a version of the algorithm proposed by Jones and 
      Smith in [6], see page $12$, corrected a mistake in it, and 
      compared the results with several alternatives.
\end{itemize}
\end{quote}
\end{comment}

In the field of Computer Graphics, Path tracing is an algorithm which 
accurately approximates global illumination in order to produce 
photo-realistic images. Path tracing has traditionally been known to 
trade speed for image quality. This is due to the lengthy process of accurately 
finding each pixels colour, whereby many light rays are 
fired through each pixel into scene, then directions for each ray are 
continually sampled until it intersects with a light source. Due to 
this, a variety of Importance sampling algorithms have been designed 
to avoid sampling directions which lead to rays contributing no light 
to the rendered image. The paths formed by sampling rays in these 
directions are known as zero contribution light paths. By not sampling 
zero contribution light paths, it is possible to significantly reduce 
the noise in rendered images using the same number of sampled rays per 
pixel in path tracing\\

Recently a Temporal Difference learning method was used by Nvidia to
achieve impressive results in Importance sampling within a Path tracer.
The algorithm essentially learns which directions light is coming from for
a given point in the scene. It then uses importance sampling to favour shooting
rays stored in those directions, reducing the number of zero contribution light paths
sampled. With this success, there is plenty of potential to experiment with
other Temporal Difference learning methods, particularly Deep Q-Learning.
It is also important to assess both of these methods on their ability to 
accurately approximate Global Illumination to produce photo-realistic images.
From this, my goal is to investigate the ability of two different temporal
difference learning algorithms ability to reduce the number of zero contribution
light paths in path tracing, whilst still accurately approximating global illumination.
More specifically, the first temporal difference learning method will be that proposed
by Nvidia, and the second will be my designed Neural-Q path tracing algorithm.
I will be comparing these two methods in order to test the following hypothesis:

\begin{quote}
The Neural-Q  path tracer is further able to reduce
the number of zero contribution light paths than an Expected SARSA
Path tracer proposed by Nvidia, whilst still accurately simulating Global Illumination.
\end{quote}

\noindent
\textbf{Outcomes}
\begin{itemize}
\item Which is better able to reduce the number of zero contribution light paths expected SARSA or Deep Q-learning
\item Can Expected SARSA learning handle multiple lights well in a scene \& deep q-learning
\end{itemize}

\noindent
\textbf{Main areas of work}
\begin{itemize}
\item I have written $x$ lines of code to build a  Path tracing engine from scratch which supports a variety of GPU accelerated Path tracing algorithms I have experimented with.

\item I have spent $x$ hours researching into the field of efficient light transport simulation for ray-tracing techniques.

\item I have spent $x$ hours researching into Reinforcement learning, particularly Temporal Difference learning and Deep Reinforcement learning, neither of which I have been taught before.

\item I spent $x$ hours implementing and validating the on-line Expected SARSA Path tracing algorithm proposed by Nvidia, which required me to implement the Irradiance Volume data structure as a prerequisite.

\item I have spent $x$ hours designing, implementing and analysing my own on-line Deep Q-learning Path tracing algorithm, along with a neural network architecture
designed for the algorithm.

\end{itemize}

\end{document}