\documentclass[conference]{IEEEtran}
\IEEEoverridecommandlockouts
% The preceding line is only needed to identify funding in the first footnote. If that is unneeded, please comment it out.
\usepackage{cite}
\usepackage{amsmath,amssymb,amsfonts}
\usepackage{algorithmic}
\usepackage{textcomp}
\usepackage{epsfig,endnotes,listings, tabulary, graphicx, tabularx}
\usepackage{xcolor}
\usepackage{hyperref}
\def\BibTeX{{\rm B\kern-.05em{\sc i\kern-.025em b}\kern-.08em
    T\kern-.1667em\lower.7ex\hbox{E}\kern-.125emX}}
\begin{document}

\title{Reinforcement Light Rays Path-Tracer Progress Report}

\author{\IEEEauthorblockN{Callum Pearce}\\ cp15571@my.bristol.ac.uk }

\maketitle

\begin{abstract}
This document contains my weekly progress of my thesis as part of my 4th year masters unit COMSM0111 in 2018/19. It is mainly useful to myself for reflecting on the decisions I have made throughout the project and why I chose them.  However, I also hope it gives a detailed overview of how the project evolved with time for any other reader.
\end{abstract}


\section{Introduction}
Each section of this document describes a single week of work. For each section I have decided to include the following break down:

\begin{itemize}

\item  \textbf{Goals}: A few set goals I aimed to achieve in that week and the motivation behind them.

\item \textbf{Research/Implementation Details}: What was done, and how it was achieved.

\item \textbf{Resources}: Describes what resources were notably helpful during the week for research/implementation details.

\item \textbf{Reflection}: What I believed went well in the week, what to avoid in the future, and were the goal outlined achieved? Finally, if necessary, what has changed for the project as a whole?

\end{itemize}

\section*{Week 1}

Building off dome preliminary research I decided I had to build a ray-tracer in order to start any work with my project. This would be a good way of refreshing the basics of computer graphics.

\subsection{Goals}
\begin{enumerate}
\item Build a basic ray-tracer from scratch using only SDL and GLM as external libraries
\end{enumerate}

\subsection{Research/Implementation Goals}
This week was fairly simple in terms of implementation. I mainly based my work on the ray-tracer I built with my project partner in my 3rd year of university. I used a similar project structure and followed the lab-sheets from  \verb|COMS30015| by Carl Henrik Ek. By the end of the week I had built a ray-tracer which simulated the following in real-time:

\begin{itemize}
\item Constructing surfaces from triangle primitives and projecting them onto a 2D pixel plane for camera viewing via ray-tracing.
\item Supports camera movement
\item Direct illumination 
\end{itemize}

\subsection{Resources}
As mentioned the main resources used were those provided for \verb|COMS30015| by Carl Henrik Ek in 2017. They gave me a good refresher on all the core concepts of ray-tracing and the mathematics behind it. I have based my rendering on the Cornell Box scene which is a classical scene for testing computer graphical renderings. 

\subsection{Reflection}
I met my goal this week and built a well designed code-base to go with it.

\section*{Week 2}
With a basic ray-tracer up and running, it was then time to add global illumination (indirect lighting) into the rendered scene and other features to make the basic ray-tracer complete for my purposes.

\subsection{Goals}
\begin{enumerate}
\item Implement Monte-Carlo global illumination within the ray-tracing pipeline (to go alongside direct light. It is Monte-Carlo global illumination I am planning to base my reinforcement learning technique presented by NVIDIA \cite{dahm2017learning} on.

\item Create an object loader for the scene to test rendering in different scenes. I need a scene which has very low light levels in certain parts of the scene in order to show how reinforcement learning reduces noise in these areas.
\end{enumerate}

\subsection{Research/Implementation Goals}
My implementation worked as follows; for every pixel in the image, calculate the colour by finding the direct light at that point combined with the indirect light. Where indirect light was sampled by shooting a ray into the scene and scattering it, then recursively finding the illumination at these points (via Monte Carlo).

For the object loader, I read in all vertices and then built triangles to form the defined surfaces in the file by using fan-triangulation. These triangles would be built into the scene by a call to the script with the file name of the \verb|.obj| file to load.

I also introduced \verb|openmp| to the project to speed things up by parallelising the ray-tracers pixel painting loop. 

\subsection{Resources}
ScratchPixel 2.0 \cite{scratch-pixel} provided an excellent description of Monte Carlo global illumination for a ray-tracer. As for the object loader, opengl gave a great tutorial for simple processing of \verb|.obj| files \cite{opengl-obj}. 

\subsection{Reflection}
Monte Carlo global illumination was available for a custom scene as I had set out to achieve in this week. I also introduced an object loader which allows me to load in a scene of my choice, which will become especially useful when comparing different methods later on in the coursework. However, due to the introduction of global illumination the ray-tracer is far slower and takes a significant amount of time (nearly a day) to render a high quality image.

\section*{Week 3}

\subsection{Goals}
\begin{enumerate}
\item 
\end{enumerate}

\subsection{Research/Implementation Goals}

\subsection{Resources}

\subsection{Reflection}


\section*{Week 4}

\subsection{Goals}
\begin{enumerate}
\item 
\end{enumerate}

\subsection{Research/Implementation Goals}

\subsection{Resources}

\subsection{Reflection}


\bibliographystyle{plain}
\bibliography{weekly-summaries}



\end{document}
