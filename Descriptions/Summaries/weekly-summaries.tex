\documentclass[conference]{IEEEtran}
\IEEEoverridecommandlockouts
% The preceding line is only needed to identify funding in the first footnote. If that is unneeded, please comment it out.
\usepackage{cite}
\usepackage{amsmath,amssymb,amsfonts}
\usepackage{algorithmic}
\usepackage{graphicx}
\usepackage{textcomp}
\usepackage{xcolor}
\def\BibTeX{{\rm B\kern-.05em{\sc i\kern-.025em b}\kern-.08em
    T\kern-.1667em\lower.7ex\hbox{E}\kern-.125emX}}
\begin{document}

\title{Reinforcement Light Rays Path-Tracer Progress Report}

\author{\IEEEauthorblockN{Callum Pearce}\\ cp15571@my.bristol.ac.uk }

\maketitle

\begin{abstract}
This document contains my weekly progress of my thesis as part of my 4th year masters unit COMSM0111 in 2018/19. It is mainly useful to myself for reflecting on the decisions I have made throughout the project and why I chose them.  However, I also hope it gives a detailed overview of how the project evolved with time for any other reader.
\end{abstract}


\section{Introduction}
Each section of this document describes a single week of work. For each section I have decided to include the following break down:

\begin{itemize}

\item  \textbf{Goals}: A few set goals I aimed to achieve in that week and the motivation behind them.

\item \textbf{Research/Implementation Details}: What was done, and how it was achieved.

\item \textbf{Resources}: Describes what resources were notably helpful during the week for research/implementation details.

\item \textbf{Reflection}: What I believed went well in the week, what to avoid in the future, and were the goal outlined achieved? Finally, if necessary, what has changed for the project as a whole?

\end{itemize}

\section*{Week 1}

Building off dome preliminary research I decided I had to build a ray-tracer in order to start any work with my project. This would be a good way of refreshing the basics of computer graphics.

\subsection{Goals}
\begin{enumerate}
\item Build a basic ray-tracer from scratch using only SDL and GLM as external libraries
\end{enumerate}

\subsection{Research/Implementation Goals}
This week was fairly simple in terms of implementation. I mainly based my work on the ray-tracer I built with my project partner in my 3rd year of university. I used a similar project structure and followed the lab-sheets from  \verb|COMS30015| by Carl Henrik Ek. By the end of the week I had built a ray-tracer which simulated the following in real-time:

\begin{itemize}
\item Constructing surfaces from triangle primitives and projecting them onto a 2D pixel plane for camera viewing via ray-tracing.
\item Supports camera movement
\item Direct illumination 
\end{itemize}

\subsection{Resources}
As mentioned the main resources used were those provided for \verb|COMS30015| by Carl Henrik Ek in 2017. They gave me a good refresher on all the core concepts of ray-tracing and the mathematics behind it.

\subsection{Reflection}
I met my goal this week and built a well designed code-base to go with it.

\section*{Week 2}

\subsection{Goals}
\begin{enumerate}
\item
\end{enumerate}

\subsection{Research/Implementation Goals}

\subsection{Resources}

\subsection{Reflection}


\section*{Week 3}

\subsection{Goals}
\begin{enumerate}
\item 
\end{enumerate}

\subsection{Research/Implementation Goals}

\subsection{Resources}

\subsection{Reflection}


\section*{Week 4}

\subsection{Goals}
\begin{enumerate}
\item 
\end{enumerate}

\subsection{Research/Implementation Goals}

\subsection{Resources}

\subsection{Reflection}


\bibliographystyle{plain}
\bibliography{weekly-summaries}



\end{document}
